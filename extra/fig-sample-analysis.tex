\documentclass[border=0.5cm]{standalone}

\usepackage[latin1]{inputenc}
\usepackage{tikz}
\usetikzlibrary{shapes, shapes.misc, arrows, positioning}
\usetikzlibrary{backgrounds,fit,calc}

\begin{document}
\pagestyle{empty}

\tikzset{
  background1/.style={
    draw,
    %fill=yellow!30,
    align=left
  }}


%% block styles
\tikzstyle{block} = [rectangle, draw=gray, fill=gray!40,
minimum width=5em, text centered, rounded corners, minimum height=4em]

\tikzstyle{inputblock} = [rectangle, draw=gray, fill=blue!20,  
 text width=5em, text centered, rounded corners, minimum height=4em]

\tikzstyle{outputblock} = [rectangle, draw=gray, fill=blue!20,  
 text width=5em, text centered, rounded corners, minimum height=4em]
 
\tikzstyle{titleblock} = [align = right]

\tikzstyle{pipeblock} = [rectangle, draw=gray, fill=blue!5,
minimum width=5em, text centered, rounded corners, minimum height=4em]

\tikzstyle{datablock} = [draw=gray, chamfered rectangle, chamfered rectangle corners=north east, minimum height=2em]

\tikzstyle{fileblock} = [draw=gray, fill=gray!5, chamfered rectangle, chamfered rectangle corners=north east, minimum height=2em]


\tikzstyle{line} = [draw, -latex']
\tikzstyle{blank} = [node distance=2cm]

\newcommand{\extendnode}[1]{
  (#1)
  ($(current bounding box.north east)!(#1)!(current bounding box.south east)$)
  ($(current bounding box.north west)!(#1)!(current bounding box.south west)$)
}


\begin{tikzpicture}[%
    >=triangle 60,
    % Global setup of box spacing
    node distance=1.5cm and 3cm, auto, 
    % Default linetype for connecting boxes
    every join/.style={norm}]

    %% Draw nodes

    % single vcf
    
    \node [datablock] (reference) {GRCh38};
    
    \node [inputblock, right= of reference] (fastq) {fastq};

    \node [pipeblock, below= of fastq] (bam1) {.bam file};

    \node [pipeblock, below= of bam1, align=center] (bam2) {processed \\ .bam file};    

    \node [datablock, right= of fastq] (rawReadsQc) {reads QC};

    \node [datablock, right= of bam2] (recal) { recalibration table };

    \node [pipeblock, below= of bam2, align=center] (bam3) { calibrated \\ .bam file };

    \node [outputblock, below= of bam3] (vcf1) { .gvcf file };

    \node [titleblock, left= of vcf1, xshift = 0cm] (alignment) {sample analysis pipeline};



    %% Draw edges

    % \path [line] (reference) edge 
    % % node [above, red, align=center]{BaseRecalibrator}
    % % node [below, align=center]{ recalibration }
    % (bwa);
    
    \path [line] (fastq) edge
    node[above, red, align=center] {fastqc}
    node[below, align=left] {QC}
    (rawReadsQc);

    \path [line] (fastq) edge
    node[left, red, align=center](bwa) {bwa \\ samblaster \\ sambamba}
    node[right, align=left] {read alignment \\ sort}
    (bam1);

    \path [line] (reference) |-
    % node [left, red, align=left]{test}
    (bwa);
    
   \path [line] (bam1) edge 
    node [left, red, align=center]{MarkDuplicates \\ sambamba}
    node [right, align=center]{process \\ sort}
    (bam2);

    \path [line] (bam2) edge 
    node [above, red, align=center]{BaseRecalibrator}
    %node [below, align=center]{ recalibration }
    (recal);

    \path [line] (bam2) edge 
    node [left, red, align=center]{ ApplyBQSR }
    node [right, align=center](bqsr){ recalibration }
    (bam3);
    
    \path [line] (bam3) edge 
    node [left, red, align=center]{HaplotypeCaller}
    node [right, align=center](vc){ variant calling}
    (vcf1);

    \path [line] (recal) |- 
    (bqsr);


    \begin{pgfonlayer}{background}
      \path [use as bounding box] (current bounding box.north west) (current bounding box.south east); % Freeze current bounding box
      \node [fit={\extendnode{fastq} (vcf1)}, background1] {};
  \end{pgfonlayer}

\end{tikzpicture}

\end{document}

%%% Local Variables:
%%% mode: latex
%%% TeX-master: t
%%% End:
