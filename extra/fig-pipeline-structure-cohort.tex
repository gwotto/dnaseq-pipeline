\documentclass[border=0.5cm]{standalone}

\usepackage[latin1]{inputenc}
\usepackage{tikz}
\usetikzlibrary{shapes, shapes.misc, arrows, positioning}
\usetikzlibrary{backgrounds,fit,calc}

\begin{document}
\pagestyle{empty}

\tikzset{
  background1/.style={
    draw,
    %fill=yellow!30,
    align=left
  }}


%% block styles

\tikzstyle{processblock} = [rectangle, draw=gray, fill=blue!5,
minimum width=5em, text centered, rounded corners, minimum height=4em]

\tikzstyle{datablock} = [draw=gray, chamfered rectangle, chamfered rectangle corners=north east, minimum height=2em]


\tikzstyle{line} = [draw, -latex']



\begin{tikzpicture}

  %% Draw nodes

  
  \node [processblock] (init) {initialize\_cohort\_analysis.py};

    \node [datablock, left= of init] (config) {config.yml};

  \node [datablock, right= of init] (samples) {\begin{tabular}{l l}cohort & sample \\
      \hline                                             
      cohort-1 & sample-1 \\
      cohort-1 & sample-2 \\
      cohort-2 & sample-3 \\
    \vdots & \vdots\end{tabular}};

    \node [processblock, below= of init] (pipeline2) {cohort\_analysis.py};

    \node [processblock, left= of pipeline2] (pipeline1) {cohort\_analysis.py};

    \node [processblock, right= of pipeline2] (pipeline3) {cohort\_analysis.py};


    
    \path [line] (config) edge
    (init);

    \path [line] (samples) edge
    (init);

    \path [line] (init) edge
    node[left, align=center](cohort1) {cohort-1}
    (pipeline1);

    \path [line] (init) edge
    node[left, align=center](cohort2) {cohort-2}
    (pipeline2);

    \path [line] (init) edge
    node[left, align=center](cohort3) {cohort-3}
    (pipeline3);

    
  \end{tikzpicture}

\end{document}

%%% Local Variables:
%%% mode: latex
%%% TeX-master: t
%%% End:
